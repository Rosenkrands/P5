
In this chapter we wish to introduce methods to create a model with parameters that best describe a given dataset.
First the multiple linear regression model will be introduced.

This chapter is based on \cite{MadsenThyregod2011} and \cite{Wooldridge2012}.

\section{The Multiple Linear Regression Model}
This project will use multiple linear regression to create a model for pricing apartments. The multiple linear regression model will therefore be introduced. The multiple linear regression model with $k+1$ parameters for observation $y_i$ is
\begin{align}\label{eq:multiple_linear_regression}
  y_i = \beta_0 + \beta_1 x_{i1} + \ldots + \beta_k x_{ik} + \varepsilon_i.
\end{align}
Here $\beta_0, \ldots, \beta_k$ are the coefficients for the explanatory variables $\textbf{x}_i = (1, x_{i1}, \ldots, x_{ik})$ and $\varepsilon_i$ is the unobserved error term.

With multiple data points $y_i$ for $i = 1, \ldots, n$ the model can be represented using matrix notation
\begin{align}\label{eq:multiple_linear_regression_model}
    \mathbf{y} = \mathbf{X} \boldsymbol{\beta} + \boldsymbol{\varepsilon}.
\end{align}
Writing the matrices out in full, we get
\begin{align}
  \begin{bmatrix}
    y_1 \\ y_2 \\ \vdots \\ y_n \\
  \end{bmatrix}
  =
  \begin{bmatrix}
    1 & x_{11} & \cdots & x_{1k} \\
    1 & x_{21} & \cdots & x_{2k} \\ \vdots & \vdots & \ddots & \vdots \\ 1 & x_{n1} & \cdots & x_{nk} \\
  \end{bmatrix}
  \begin{bmatrix}
    \beta_0 \\ \beta_1 \\ \vdots \\ \beta_k \\
  \end{bmatrix} +
  \begin{bmatrix}
    \varepsilon_1 \\ \varepsilon_2 \\ \vdots \\ \varepsilon_n \\
  \end{bmatrix}.
\end{align}
Note that the first column in $\mathbf{X}$ consists of all 1's, since $\beta_{0}$ is the intercept and therefore has coefficient $1$. $\textbf{X}$ is often referred to as the design matrix. 

When $\textbf{y}$, $\textbf{X}$ and the distribution of $\boldsymbol{\varepsilon}$ are known, it is possible to estimate the unknown parameters $\boldsymbol{\beta}$. 
In order to determine these parameters, $\boldsymbol{\beta}$, we will use the ordinary least squares (OLS) estimate, though there are other methods of estimating parameters for this model.