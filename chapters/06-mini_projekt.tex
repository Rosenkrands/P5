\newpage
\section*{Mini-project}
We have $Y_1,...,Y_n$ as underlying random variables for observations/data/responses $y_1,...,y_n$.
Specifically, we let $Y_1,...,Y_n$ be independent normally distributed random with variance $1$, and let $\mu_i= E[Y_i]$ be the mean of $Y_i, i= 1,...,n$.
Furthermore, we let $x_1,...,x_n$ be given numbers; when the distribution of $(Y_1,...,Y_n)$ depends on $(x_1,...,x_n)$ we also call $(Y_1,...,Y_n)$ the dependent variables and $(x_1,...,x_n)$ the explanatory variables (or the independent variables, though this may be misleading terminology).

\subsection*{Exercise 2.1 The First Model}

Assume that
\begin{align*}
    \mu_i = \alpha, \quad i = 1, \ldots,n
\end{align*}
where $\alpha$ is an unknown real parameter. 

\begin{enumerate}
    \item Specify the log likelihood function, the score function, the observed information matrix and the Fisher information matrix
\end{enumerate}

\subsubsection{Log likelihood function}
First we find the likehood function, which here is the product of standard normal distribution. The pdf is 

\begin{align*}
   L(Y;\mu) &= \prod_{i=1}^n \left[ \frac{1}{\sigma \sqrt{2 \pi}}exp\left(-\frac{(y_i -\mu_i)^2}{2 \sigma^2}\right) \right]\\
\end{align*}

Now we take log to the above equation to obtain the log likelihood function

\begin{align*}
   \ell(Y;\alpha) &= log \left( \prod_{i=1}^n \left[ \frac{1}{\sigma \sqrt{2 \pi}}exp\left(-\frac{(y_i -\mu_i)^2}{2 \sigma^2}\right) \right] \right)\\
   &= \sum_{i = 1}^n \left[ log\left( \frac{1}{\sigma \sqrt{2 \pi}}exp\left[-\frac{(y_i - \mu_i)^2}{2 \sigma^2}\right] \right) \right]\\
   &= \sum_{i = 1}^n \left[ log(1) - log(\sqrt{2 \pi}) - \frac{(y_i - \alpha)^2}{2} \right]\\
   &= \sum_{i = 1}^n \left[- log\left( \sqrt{2 \pi}\right) - \left(\frac{y_i^2 + \alpha^2 - 2y_i\alpha}{2}\right) \right]\\
   &= \sum_{i = 1}^n \left[y_i \alpha - log\left( \sqrt{2 \pi}\right) - \left( \frac{y_i^2 + \alpha^2}{2} \right) \right]
\end{align*}

\subsubsection{Score function}

The score function is the log likelihood function differentiated with respect to its parameter, thus

\begin{align*}
    S\left( Y; \alpha \right) &= \ell'(Y; \alpha)\\
    &= \sum_{i=1}^n \left[ y_i + 0 - \alpha \right]\\
    &= \sum_{i=1}^n \left[ y_i - \alpha \right].
\end{align*}

This is the score function. 

\subsubsection{Observed information}

We now find the observed information, which is the score function differentiated and taken minus to

\begin{align*}
    j\left( Y; \alpha \right) &= - S\left(Y; \alpha \right)\\
    &=  - \left( \sum_{i = 1}^n -1\right) = n.
\end{align*}

\subsubsection{Fisher information matrix}

The Fisher information matrix is the expected value of observed information

\begin{align*}
    i\left(\alpha \right) &= E\left[j(Y;\alpha)\right]\\
    &= n.
\end{align*}

\begin{enumerate}[resume]
    \item  Find the MLE $\hat{\alpha}$.
\end{enumerate}

The MLE is defined as

\begin{align*}
    L(\hat{Y}; \alpha) &= sup L(Y; \alpha)
\end{align*}

and it is found by setting the score function equal to $0$

\begin{align*}
    S\left(Y; \alpha \right) &= 0\\
    \sum_{i=1}^n \left[y_i - \alpha \right]&= 0\\
    n\alpha &= \sum_{i=1}^n y_i\\
    \hat{\alpha} &= \frac{1}{n} \sum_{i=1}^n y_i \\
    \hat{\alpha} &= \bar{y},
\end{align*}
where $\bar{y}$ is the mean of the observed values $y_i$ for $i = 1, \ldots n$.

\begin{enumerate}[resume]
    \item  Specify the asymptotic distribution of $\hat{\alpha}\xrightarrow \infty$. Use this to obtain an approximate $95\%$-confidence interval for $\alpha$.
\end{enumerate}

\begin{theorem}[Distribution of the ML estimator]\label{th:distribution_ml_estimator}
Let $E = \{\mathbf{y} \ : \ \text{the MLE } \hat{\theta}(\mathbf{y}) \text{ exists}\}$. 
If $A$ is a $k \times k$ matrix, $A^{1/2}$ denotes the $k \times k$ matrix such that $A = A^{1/2}\left( A^{1/2} \right)^T$.
Let $\rightarrow^\mathcal{D}$ denote convergence in distribution, $\mathbf{1}[\cdot]$ the indicator function and $I_k$ the $k \times k$ identity matrix.
Then under regularity conditions, if $\mathbf{Y} \sim f(\cdot;\theta)$ then as $n \rightarrow \infty$ we have that
\begin{enumerate}
    \item $P_\theta(Y \in E) \rightarrow 1$
    \item for any $\varepsilon > 0: \ P_\theta(Y \in E, \ \|\hat{\theta} - \theta\| \leq 1) \rightarrow 1$
    \item $1[\mathbf{y} \in E] i(\hat{\theta})^{1/2}(\hat{\theta} - \theta) \rightarrow^\mathcal{D} N_k(0, I_k)$ and $1[\mathbf{y} \in E] j(\hat{\theta})^{1/2}(\hat{\theta} - \theta) \rightarrow^\mathcal{D} N_k(0, I_k)$
    \item $1[\mathbf{y} \in E] i(\theta)^{1/2}(\hat{\theta} - \theta) \rightarrow^\mathcal{D} N_k(0, I_k)$
\end{enumerate}
\end{theorem}

% \begin{center}
% -----
% \textbf{Udledning mangler}
% -----
% \end{center}

By theorem \ref{th:distribution_ml_estimator} the asymptotic distribution of the ML estimator is given by
\begin{align*}
    \hat{\theta} \rightarrow^\mathcal{D} N\left( \theta, \boldsymbol{i}(\theta)^{-1} \right)
\end{align*}

Since $\alpha$ is one-dimensional, $k=1$, and for $0\leq a \leq 0.5$ we have the approximate $(1-a)$-confidence interval
\begin{align*}
\left[ \hat{\alpha}(\mathbf{y}) + \Phi_{a/2} \sqrt{D\left( \hat{\alpha}\left(\mathbf{y}\right)\right)}, \hat{\alpha}(\mathbf{y}) + \Phi_{1 - a/2} \sqrt{D\left( \hat{\alpha}\left(\mathbf{y}\right)\right)} \right],
\end{align*}
where $D\left(\hat{\alpha}(\mathbf{y})\right) = j\left(\hat{\alpha}(\mathbf{y})\right)^{-1}$.
For $a=0.05$ we have the approximate $95\%$-confidence interval for $\alpha$:
\begin{align*}
\left[ \bar{y} - \frac{1.96}{\sqrt{n}}, \bar{y} + \frac{1.96}{\sqrt{n}} \right]    
\end{align*}

\subsection*{Exercise 2.2 The second model}

Assume that
\begin{align*}
    \mu_i=\alpha \beta^{x_i}, \quad i=1,\ldots, n
\end{align*}
where $\alpha$ and $\beta$ are unknown real parameters. Then $x_1, \ldots,x_n$ are called explanatory variables. 

\begin{enumerate}
    \item Specify the log likelihood function, the score function, the observed information matrix and the Fisher information matrix
\end{enumerate}

\textbf{Log likelihood function}
In order to find the log likelihood function, we first find the likelihood function

\begin{align*}
   L(Y;\mu) &= \prod_{i=1}^n \frac{1}{\sigma \sqrt{2 \pi}}exp\left[-\frac{(y_i -\mu)^2}{2 \sigma^2}\right].\\
\end{align*}

Now we take log of the above equation, to obtain the loglikelihood

\begin{align*}
   \ell(Y;\alpha, \beta) &= log \left( \prod_{i=1}^n \frac{1}{\sigma \sqrt{2 \pi}}exp\left[-\frac{(y_i -\mu)^2}{2 \sigma^2}\right] \right)\\
   &= \sum_{i = 1}^n \left[ y_i \alpha \beta^{x_i} - log\left( \sqrt{2 \pi}\right) - \left( \frac{y_i^2 + (\alpha \beta^{x_i})^2}{2} \right) \right]\\
   &=\sum_{i = 1}^n \left[ y_i \alpha \beta^{x_i} - log\left( \sqrt{2 \pi}\right) - \left( \frac{y_i^2 + (\alpha^2 \beta^{2x_i})}{2} \right) \right].\\
\end{align*}

\textbf{Score function}

\begin{align*}
    S\left( Y; \alpha, \beta \right) = \begin{bmatrix} S(Y;\alpha) \\ S(Y; \beta) \end{bmatrix} = \begin{bmatrix} \sum_{i=1}^n y_i \beta^{x_i} - \alpha\beta^{2x_i} \\ \sum_{i = 1}^n y_i\alpha x_i \beta^{x_i - 1} - \alpha^2 x_i \beta^{2x_i - 1} \end{bmatrix}.
\end{align*}

Here the Score function is a vector, because it has more than one variable. 

\subsubsection{Observed information}

We now find the observed information as before, but because of the parameters, it is now a matrix

\begin{align*}
    j\left( Y; \alpha ,\beta \right) = \begin{bmatrix} 
    \sum_{i = 1}^n \beta^{2x_i} 
    &
    \sum_{i = 1}^n 2\alpha x_i \beta^{2x_i - 1} - y_ix_i\beta^{x_i - 1}
    \\
    \sum_{i = 1}^n 2\alpha x_i \beta^{2x_i - 1} - y_ix_i\beta^{x_i - 1} 
    &
    \sum_{i = 1}^n \alpha^2(2x_i - 1)x_i\beta^{2x_i - 2} - y_i \alpha x_i(x_i - 1) \beta^{x_i - 2}
    \end{bmatrix}.
\end{align*}

\subsubsection{Fisher information matrix}

The Fisher information matrix is the expected value of observed information

\begin{align*}
    i\left(\alpha \beta^{x_i} \right) &= E\left[j(Y;\alpha ,\beta)\right]\\
    &= \begin{bmatrix} 
    \sum_{i = 1}^n \beta^{2x_i} 
    &
    \sum_{i = 1}^n 2\alpha x_i \beta^{2x_i - 1} - \alpha\beta^{x_i}x_i\beta^{x_i - 1}
    \\
    \sum_{i = 1}^n 2\alpha x_i \beta^{2x_i - 1} - \alpha\beta^{x_i}x_i\beta^{x_i - 1}
    &
    \sum_{i = 1}^n \alpha^2(2x_i - 1)x_i\beta^{2x_i - 2} - \alpha \beta^{x_i} \alpha x_i(x_i - 1) \beta^{x_i - 2}
    \end{bmatrix} \\
    &= \begin{bmatrix} 
    \sum_{i = 1}^n \beta^{2x_i}  
    &
     \sum_{i = 1}^n 2\alpha x_i \beta^{2x_i - 1} - \alpha\beta^{2x_i-1}x_i
    \\
    \sum_{i = 1}^n 2\alpha x_i \beta^{2x_i - 1} - \alpha\beta^{2x_i-1}x_i
    &
    \sum_{i = 1}^n \alpha^2(2x_i - 1)x_i\beta^{2x_i - 2} - \alpha \beta^{2x_i - 2} \alpha x_i(x_i - 1)
    \end{bmatrix}\\
    &= \begin{bmatrix} 
    \sum_{i = 1}^n \beta^{2x_i}  
    &
     \sum_{i = 1}^n  \alpha\beta^{2x_i-1}x_i
    \\
    \sum_{i = 1}^n  \alpha\beta^{2x_i-1}x_i
    &
    \sum_{i = 1}^n n\alpha^2 x_i^2 \beta^{2x_i - 2}
    \end{bmatrix}.  
\end{align*}

\begin{enumerate}[resume]
    \item Discuss how you will find the MLE in the following cases: 
    \begin{enumerate}[label = (\alph*)]
        \item By using the optim command in R
    \end{enumerate}
The optim function in R takes as input, a vector containing initial values and a objective function.
The function then uses as default the ``Nelder-Mead'' procedure for minimizing the objective function. This is a procedure that uses only the function values and is therefore useful for dealing with non-differentiable functions.

When dealing with differentiable functions the performance of the optim function can be improved as the function can now use other methods such as BFGS. 
This is a method the makes use of the gradient of the objective function.
It can therefore find a minimum of the function in fewer iterations.

As the optim function takes an initial value, the point at which the function terminates will vary depending on the initial value.
For the function given by
\begin{align*}
    f(x,y) = x^2 + y^2
\end{align*}
the initial value makes no difference as the function has only one minimum, namely at $(0,0)$.
Consider instead the function
\begin{align*}
    g(x,y) = sin(x) + cos(y)
\end{align*}
as this function have an indefinite amount of minimums, the minimum returned by the function will depend on the initial value given.
    \begin{enumerate}[label = (\alph*), resume]
        \item First fix $\beta$ and find the MLE of $\alpha$ in term of $\beta$, denoted $\hat{\alpha}(\beta)$, next find the MLE of $\beta$ and hence that of $\alpha$-the likelihood function with $\alpha$ replaced by $\hat{\alpha}(\beta)$ is called the profile likelihood of $\beta$ (or the partially maximized likelihood function) and is denoted $\hat{L}(\beta)$. Specify the Newton-Raphson procedure for finding the MLE of $\beta$ by maximizing $\hat{L}(\beta)$. Will the procedure converge and return that MLE?
    \end{enumerate}
\end{enumerate}


First we find the MLE of $\alpha$ for fixed $\beta$
\begin{align*}
    S\left( Y; \alpha \right) = 0 \\
    \sum_{i=1}^{n} \left[ y_i \beta^{x_i} - \alpha \beta^{2 x_i} = 0 \right] \\
    \sum_{i=1}^{n} y_i \beta^{x_i} = \alpha \sum_{i=1}^{n} \beta^{2 x_i} \\
    \hat{\alpha} = \frac{\sum_{i=1}^{n} y_i \beta^{x_i}}{\sum_{i=1}^{n} \beta^{2x_i}}.
\end{align*}
This estimate is substituted into the score function for $\beta$, to obtain the profile score function of $\beta$, $\hat{S}(\beta)$
\begin{align*}
    S\left( Y; \beta \right) &= \sum_{i=1}^n \left[ y_i x_i \alpha \beta^{x_i - 1} - \alpha^2 x_i \beta^{2 x_i - 1} \right]\\
     &= \sum_{i=1}^n \left[ y_i x_i \beta^{x_i - 1} \left( \dfrac{\sum_{i=1}^n y_i\beta^{x_i}}{\sum_{i=1}^n \beta^{2x_i}}\right) - x_i\beta^{2x_i-1} \left( \dfrac{\sum_{i=1}^n y_i\beta^{x_i}}{\sum_{i=1}^n \beta^{2x_i}}\right)^2 \right].
\end{align*}
Using Taylor expansion and isolating $\beta$ in this expression, we obtain
\begin{align*}
    S_\beta (\beta_i, y_i x) &\approx S_\beta(\beta^*) - i(\beta^*) (\hat{\beta} - \beta^*) = 0 \quad \Rightarrow \quad
    \hat{\beta} = \beta^* + i(\beta^*)^{-1} S_\beta (\beta^*).
\end{align*}
Newton-Raphson finds the roots and thus in our case the maximum. 
The procedure is used to estimate the MLE in cases where $\beta$ cannot be calculated analytically but must be found numerically, therefore the returned MLE is an estimate, that maximizes the log likelihood function.  
As the procedure is sensitive to the initial value given, whether or not it will converge depends.

\begin{enumerate}[resume]
    \item Specify the asymptotic distribution of the MLE $(\hat{\alpha}, \hat{\beta})$ as $n \rightarrow \infty$.
    Use this to obtain approximate 95\%-confidence intervals for $\alpha$ and $\beta$, respectively, and an 95\%-confidence region for $(\alpha, \beta)$
\end{enumerate}

By theorem 2.4 as the sample size tends to infinity the distribution of the ML estimator converges to the normal distribution.
The 95\%-confidence interval for $\alpha$ becomes
\begin{align*}
    \left[ \left( \frac{\sum_{i=1}^n y_i \beta^{x_i}}{\sum_{i=1}^n \beta^{2x_i}}\right) \pm 1.96 \sqrt{D_{11}(Y; \alpha, \beta)} \right]
\end{align*}
where $D = j^{-1}$.
The 95\%-confidence interval for $\beta$ becomes
\begin{align*}
    \left[ \hat{\beta}(y) \pm 1.96 \sqrt{D_{22}(Y; \alpha, \beta)} \right]
\end{align*}
\begin{definition}
Suppose that for a given number $a \in (0,1)$ and for each $y \in \mathcal{Y}$, we have a specified subset $A(y)$ of the parameter space $\Theta^k$ such that
\begin{align*}
    P_\theta(\theta \in A(Y)) = 1 - a \qquad \text{for all } \theta \in \Theta^k.
\end{align*}
Then we call $A(y)$ an (\textit{exact}) $(1 - a)$-\textit{confidence region for} $\theta$.
\end{definition}
One way to define the confidence region is based on the likelihood ratio test.
In our case $A(y)$ becomes
\begin{align*}
    A(y) = \{ (\alpha, \beta) \in \mathbb{R}^2 \ : \ -2 \: \ell_{norm}((\alpha, \beta), Y) \leq \chi^2_{1 - a}(k))\}.
\end{align*}

\begin{enumerate}[resume]
    \item Specify methods for testing the hypothesis $\beta = 1$, i.e., when the model reduces to model 1.
\end{enumerate}
One way to test the hypothesis is using Wilk's likelihood ratio test.

\subsection*{2.3 The third model}
Assume that
\begin{align*}
    \mu_i=\alpha \beta^{x_i}+\gamma, \quad i=1,\vdots n,
\end{align*}
where $\alpha$, $\beta$ and $\gamma$ are unknowwn parameters.
\begin{enumerate}
    \item Specify the log likelihood function, the score function, the observed information matrix and the Fisher information matrix
\end{enumerate}

\subsubsection{Log likelihood function}

In order to find the log likelihood function, we first find the likelihood function

\begin{align*}
   L(Y;\mu) &= \prod_{i=1}^n \left[\frac{1}{\sigma \sqrt{2 \pi}}exp\left(-\frac{(y_i -\mu)^2}{2 \sigma^2}\right) \right].\\
\end{align*}

Now we take log of the above equation, to obtain the log likelihood

\begin{align*}
   \ell(Y;\alpha, \beta, \gamma) &= log \left( \prod_{i=1}^n \left[\frac{1}{\sigma \sqrt{2 \pi}}exp\left(-\frac{(y_i -\mu)^2}{2 \sigma^2}\right) \right] \right)\\
   &= \sum_{i = 1}^n \left[ y_i (\alpha \beta^{x_i}+\gamma) - log\left( \sqrt{2 \pi}\right) - \left( \frac{y_i^2 + (\alpha \beta^{x_i}+\gamma)^2}{2} \right) \right] \\
   &=\sum_{i = 1}^n \left[ y_i \alpha \beta^{x_i} +y_i\gamma - log\left( \sqrt{2 \pi}\right) - \left( \frac{y_i^2 + (\alpha^2 \beta^{2x_i}) + \gamma^2 + 2\alpha\beta^{x_i}\gamma}{2}\right) \right].\\
\end{align*}

\subsection{Score function}

\begin{align*}
    S(y;\alpha,\beta,\gamma)=
    \begin{bmatrix}
    &S(y;\alpha) \\
    &S(y;\beta) \\
    &S(y;\gamma)
    \end{bmatrix} 
\end{align*}

where 

\begin{align*}
    S(y;\alpha) &=\sum_{i=1}^n \left[ y_i\beta^{x_i} - \frac{2\alpha\beta{2x_i}+2\beta^{x_i}\gamma}{2} \right] \\
    &= \sum_{i=1}^n \left[ y_i\beta^{x_i}-\alpha\beta^{2x_1}-\beta^{x_i}\gamma \right]
\end{align*}

\begin{align*}
    S(y;\beta) &= \left[ \sum_{i=1}^n x_i y_i\alpha\beta^{x_i - 1} - \frac{2x_i\alpha^2\beta^{2x_i -1}+2x_i\alpha\beta^{x_i-1}\gamma}{2} \right] \\
    &= \sum_{i=1}^n \left[ x_i y_i\alpha\beta^{x_i-1} - x_i\alpha^2\beta^{2x_i-1} - x_i\alpha\beta^{x_i-1}\gamma \right]
\end{align*}

\begin{align*}
    S(y;\gamma) &= \sum_{i=1}^n \left[ y_i - \frac{2\gamma + 2\alpha\beta^{x_i}}{2} \right]\\
    &= \sum_{i=1}^n \left[ y_i-\gamma-\alpha\beta^{x_i} \right]
\end{align*}

Observed information

\begin{align*}
    &j(Y;\alpha,\beta,\gamma) = \\
    &-\begin{bmatrix}
        \sum_{i=1}^n \beta^{x_i} & \sum_{i=1}^n x_iy_i\beta^{x_i-1}-2x_i\alpha\beta^{2x_i}+x_i\beta^{x_i-1}\gamma & \sum_{i=1}^n \beta^{x_i} \\
       \sum_{i=1}^n x_iy_i\beta^{x_i-1}-2x_i\alpha\beta^{2x_i}+x_i\beta^{x_i-1}\gamma & \sum_{i=1}^n x_i \alpha \beta^{x_i-2}\Big( (x_i - 1) \big( y_i  +  \gamma \big)- (2x_i-1) \alpha \beta^{x_i} \Big) & \sum_{i=1}^n x_i \alpha \beta^{x_i - 1} \\
        \sum_{i=1}^n \beta^{x_i} & \sum_{i=1}^n x_i \alpha \beta^{x_i - 1} & n
    \end{bmatrix}
\end{align*}

Fisher information

\begin{align*}
    &i(Y;\alpha,\beta,\gamma) = \\
    &-\begin{bmatrix}
        \sum_{i=1}^n \beta^{x_i} & \sum_{i=1}^n x_i\alpha\beta^{2x_i-1}+x_i\gamma \beta^{x_i-1}-2x_i\alpha\beta^{2x_i} & \sum_{i=1}^n \beta^{x_i} \\
       \sum_{i=1}^n x_i\alpha\beta^{2x_i-1}+x_i\gamma \beta^{x_i-1}-2x_i\alpha\beta^{2x_i} & \sum_{i=1}^n x_i \alpha \beta^{x_i-2}\Big( 2x_i\gamma - x_i\alpha\beta^{x_i} \Big) & \sum_{i=1}^n x_i \alpha \beta^{x_i - 1} \\
        \sum_{i=1}^n \beta^{x_i} & \sum_{i=1}^n x_i \alpha \beta^{x_i - 1} & n
    \end{bmatrix}
\end{align*}

%Indgang 2,2 \sum_{i=1}^n x_i (x_i - 1) y_i \alpha \beta^{x_i - 2} - x_i (2 x_i -1) \alpha^2 \beta^{2x_i -2} + x_i (x_i -1) \alpha \beta^{x_i -2} \gamma