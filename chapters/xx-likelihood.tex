The aim of this chapter is to present some of the key results from likelihood theory. 
The purpose of likelihood theory is to find the statistical model, which is most compatible with observed data. 
Through maximum likelihood estimation one obtains the estimates of the parameters in a statistical model, which maximise the likelihood function. 
This parameterization of the distribution has most likely generated the data. \\


\begin{definition} [Unbiased Estimator]
\label{def:Unbiased_estmator}
Any estimator $\boldsymbol{\hat{\beta}} = \boldsymbol{\hat{\beta}}(Y)$ is said to be unbiased if $E[\boldsymbol{\hat{\beta}}] = \boldsymbol{\beta}, \ \forall \boldsymbol{\beta} \in \Theta^k$.
\end{definition}

This means, that an estimator is considered as unbiased, if it is equal to the true parameter value. 
In other words, the estimator is neither overestimated nor underestimated.

\begin{definition} [Consistent Estimator]
\label{def:consistent_estimator}
An estimator is consistent if the sequence $\boldsymbol{\beta}_n(\textbf{Y})$ of estimators for all $\boldsymbol{\beta} \in \Theta^k$ and $\epsilon > 0$,
%for the parameter $\boldsymbol{\beta}$ converges in probability to the true value $\boldsymbol{\beta}$.
\begin{align*}
    P_{\boldsymbol{\beta}}(||\hat{\boldsymbol{\beta}}(\textbf{Y}) - \boldsymbol{\beta}|| > \epsilon) \xrightarrow[n \rightarrow \infty]{P} 0
\end{align*}
Otherwise the estimator is said to be inconsistent. 
\end{definition}
In other words, the parameter is said to be consistent, if the estimate converges towards the true value. 

\begin{definition} [Minimum Mean Square Error]
\label{def:minimum_mean_square_error}
An estimator $\boldsymbol{\hat{\beta}}=\boldsymbol{\hat{\beta}}(\textbf{Y})$ is said to be \textit{uniformly minimum square error}, if
\begin{align*}
    E\big[ (\boldsymbol{\hat{\beta}}(\textbf{Y})-\boldsymbol{\beta})(\boldsymbol{\hat{\beta}}(\textbf{Y})-\boldsymbol{\beta})^T \big] \leq E\big[ (\boldsymbol{\tilde{\beta}}(\textbf{Y})-\boldsymbol{\beta})(\boldsymbol{\tilde{\beta}}(\textbf{Y})-\boldsymbol{\beta})^T \big]
\end{align*} 
for all $\boldsymbol{\beta} \in \Theta^k $ and all other estimators $\boldsymbol{\tilde{\beta}(\textbf{Y})}$.
 \end{definition}

\begin{theorem} [Cramer-Rao Inequality]
\label{th:cramerrao_inequality}
Given the paramtetric density $f_{\textbf{Y}}(\textbf{y};\boldsymbol{\beta}), \boldsymbol{\beta} \in \Theta^k$ for all observations $\textbf{Y}$.
Given certain regularity conditions, the covariance matrix of any unbiased estimator $\boldsymbol{\hat{\beta}}(\textbf{Y})$ of $\boldsymbol{\beta}$ satisfies the inequality
\begin{align} \label{eq:cramerrao_inequality}
    \var \Big[ \boldsymbol{\hat{\beta}}(\textbf{Y})\Big] \geq \textit{\textbf{i}}^{-1}(\boldsymbol{\beta})
\end{align}
where $\textit{i}(\beta)$ is \textit{the Fisher information matrix} which is defined by
\begin{align*}
    \textit{\textbf{i}}(\boldsymbol{\beta})=E \bigg[ \bigg(\frac{\partial \log f_Y(\textbf{Y};\boldsymbol{\beta})}{\partial \boldsymbol{\beta}} \bigg)\bigg(\frac{\partial \log f_Y(\textbf{Y};\boldsymbol{\beta})}{\partial \boldsymbol{\beta}}\bigg)\bigg]
\end{align*}
where $\var [\boldsymbol{\hat{\beta}}(\textbf{Y})]=E[(\boldsymbol{\hat{\beta}}(\textbf{Y})-\boldsymbol{\beta})(\boldsymbol{\hat{\beta}}(\textbf{Y})-\boldsymbol{\beta})^T]$.
\end{theorem}
I.e. regularity conditions, the theorem is subject to that $\textit{\textbf{i}}(\boldsymbol{\beta})$ is invertible for all $\boldsymbol{\beta} \in \Theta^k$ and interchanging of the of derivative and integration.
Note that the inequality in \eqref{eq:cramerrao_inequality} entails that the right hand side subtracted from the left hand side in non-negative definite.

\begin{proof}
We see that
\begin{align*}
    E\Big[\boldsymbol{\hat{\beta}}(\textbf{Y}) \frac{\partial \log f_Y(\textbf{Y};\boldsymbol{\beta})}{\partial \boldsymbol{\beta}} \Big]
    &=\int \boldsymbol{\hat{\beta}}(\textbf{y})\frac{\partial \log f_Y(\textbf{y};\boldsymbol{\beta})}{\partial \boldsymbol{\beta}}f_Y(\textbf{y};\boldsymbol{\beta}) \text{dy} \\
    &=\int \boldsymbol{\hat{\beta}}(\textbf{y}) \frac{\partial}{\partial \boldsymbol{\beta}}f_Y(\textbf{y};\boldsymbol{\beta}) \text{dy} \\
    &=\frac{\partial}{\partial \boldsymbol{\beta}} \int \boldsymbol{\hat{\beta}}(\textbf{y}) f_Y(\textbf{y};\boldsymbol{\beta}) \text{dy}
\end{align*}
where we obtain the last equation from the regularity conditions. 
Since $\boldsymbol{\hat{\beta}}(\textbf{Y})$ is unbiased, definition \ref{def:Unbiased_estmator} 
\begin{align}
    \frac{\partial}{\partial \boldsymbol{\beta}} \int \boldsymbol{\hat{\beta}}(\textbf{y}) f_Y(\textbf{y};\boldsymbol{\beta}) \text{dy} &= \frac{\partial}{\partial\boldsymbol{\beta}} E\big[ \boldsymbol{\hat{\beta}}(\textbf{Y})\big] \nonumber\\
    &= \frac{\partial}{\partial\boldsymbol{\beta}} \boldsymbol{\beta} \nonumber\\
    &= I_{k\times k} \label{eq:cramerraoe2stjerner}.
\end{align}
Furthermore, we see that
\begin{align}
    E\Big[ \frac{\partial \log f_Y(\textbf{Y};\boldsymbol{\beta})}{\partial \boldsymbol{\beta}}\Big] &= \int \frac{\partial \log f_Y(\textbf{y};\boldsymbol{\beta})}{\partial \boldsymbol{\beta}}f_Y(\textbf{y};\boldsymbol{\beta}) \text{dy} \nonumber \\
    &=\int  \frac{\partial}{\partial \boldsymbol{\beta}}f_Y(\textbf{y};\boldsymbol{\beta}) \text{dy} \nonumber \\
    &=\frac{\partial}{\partial \boldsymbol{\beta}} \int f_Y(\textbf{y};\boldsymbol{\beta}) \text{dy} \nonumber \\
    &= \textbf{0}_{1 \times k} \label{eq:cramerraoe3stjerner}.
\end{align}
From above we are able to find the covariance matrix for $\begin{bmatrix} \boldsymbol{\hat{\beta}}(\textbf{Y}) & \partial \log f_Y(\textbf{Y};\boldsymbol{\beta})/\partial \boldsymbol{\beta}) \end{bmatrix}^T$.
\begin{align*}
    \var \begin{bmatrix}  \boldsymbol{\hat{\beta}}(\textbf{Y}) \\  \partial \log f_Y(\textbf{Y};\boldsymbol{\beta})/\partial \boldsymbol{\beta}) \end{bmatrix} &= E \begin{bmatrix} \begin{pmatrix} \boldsymbol{\hat{\beta}}(\textbf{Y})-\boldsymbol{\beta} \\  \partial \log f_Y(\textbf{Y};\boldsymbol{\beta})/\partial \boldsymbol{\beta})^T\end{pmatrix} \begin{pmatrix} \boldsymbol{\hat{\beta}}(\textbf{Y}-\boldsymbol{\beta})^T &  \partial \log f_Y(\textbf{Y};\boldsymbol{\beta})/\partial \boldsymbol{\beta} \end{pmatrix}\end{bmatrix}  \\
    &= \var \begin{bmatrix} \boldsymbol{\hat{\beta}}(\textbf{Y}) & I_{k \times k}\\
    I_{k \times k} & \textbf{i}(\boldsymbol{\beta})\end{bmatrix}.
\end{align*}
Because of symmetric in \eqref{eq:cramerraoe2stjerner} and \eqref{eq:cramerraoe3stjerner} the covariance matrix is clearly non-negative definite,and we have
\begin{align*}
    0_{k \times k} & \leq \begin{bmatrix} I_{k\times k} & -\textbf{i}^{-1}(\boldsymbol{\beta})\end{bmatrix} \begin{bmatrix} \var[\boldsymbol{\hat{\beta}}(\textbf{Y})] & I_{k\times k} \\ I_{k\times k} & \textbf{i}(\boldsymbol{\beta}) \end{bmatrix}\begin{bmatrix} I_{k\times k} \\ -\textbf{i}^{-1}(\boldsymbol{\beta})\end{bmatrix} \\
    &= \begin{bmatrix} \var[\boldsymbol{\hat{\beta}}(\textbf{Y})] -\textbf{i}^{-1}(\boldsymbol{\beta}) & 0_{k \times k}\end{bmatrix} \begin{bmatrix} I_{k\times k} \\ -\textbf{i}^{-1}(\boldsymbol{\beta})\end{bmatrix} \\
    &= \var[\boldsymbol{\hat{\beta}}(\textbf{Y})] -\textbf{i}^{-1}(\boldsymbol{\beta})
\end{align*}
which establishes the Cramer-Rao inequality.
\end{proof}

\begin{definition} [Efficient Estimator]
\label{def:efficient_estimator}
An unbiased estimator is said to be efficient, if its covariance-matrix is equal to the Cramer-Rao lower bound
\end{definition}

\begin{definition} [Likelihood Function]
\label{def:likelihood_function}
Given the data $\textbf{y}$ for a parametric model with density $f_Y(\textbf{y}$ and parameter space $\Theta^k$. The likelihood function for $\boldsymbol{\beta}$ is any function of the form 
\begin{align*}
    L(\boldsymbol{\beta}; \textbf{y}) = c(y_1, y_2, \ldots, y_n)f_Y(y_1, y_2, \ldots, y_n; \boldsymbol{\beta}), 
\end{align*}
where $c(\textbf{y})>0$ does not depend on $\boldsymbol{\beta}$. 
\end{definition}

The likelihood function is only meaningful, for the terms involving the parameter, meaning that we can ignore constant terms.

It is often more convenient to work with the log-likelihood, and since the logarithm only changes the function values, and not the location of the critical points, it will always have the same maximum point as the likelihood function. 

\begin{align*}
    \ell(\boldsymbol{\beta};\textbf{y})=log(L(\boldsymbol{\beta}; \textbf{y})).
\end{align*}

\begin{example} \label{ex:model1}
We have $Y_1,...,Y_n$ as underlying random variables for observations/data/responses $y_1,...,y_n$.
Specifically, we let $Y_1,...,Y_n$ be independent normally distributed random with variance $1$, and let $\mu_i= E[Y_i]$ be the mean of $Y_i, i= 1,...,n$.
Furthermore, we let $x_1,...,x_n$ be given numbers; when the distribution of $(Y_1,...,Y_n)$ depends on $(x_1,...,x_n)$ we also call $(Y_1,...,Y_n)$ the dependent variables and $(x_1,...,x_n)$ the explanatory variables (or the independent variables, though this may be misleading terminology).
\\
Assume that
\begin{align*}
    \mu_i = \alpha, \quad i = 1, \ldots,n
\end{align*}
where $\alpha$ is an unknown real parameter. 
The likelihood function of the model is the product of standard normal distributions. The pdf is 

% \begin{align*}
%   L(\textbf{Y};\textbf{\mu}) &= \prod_{i=1}^n \left[ \frac{1}{\sigma \sqrt{2 \pi}}exp\left(-\frac{(y_i -\mu_i)^2}{2 \sigma^2}\right) \right]\\
% \end{align*}

Now we take log to the above equation to obtain the log likelihood function

\begin{align*}
   \ell(\textbf{Y};\alpha) &= log \left( \prod_{i=1}^n \left[ \frac{1}{\sigma \sqrt{2 \pi}}exp\left(-\frac{(y_i -\mu_i)^2}{2 \sigma^2}\right) \right] \right)\\
   &= \sum_{i = 1}^n \left[ log\left( \frac{1}{\sigma \sqrt{2 \pi}}exp\left[-\frac{(y_i - \mu_i)^2}{2 \sigma^2}\right] \right) \right]\\
   &= \sum_{i = 1}^n \left[ log(1) - log(\sqrt{2 \pi}) - \frac{(y_i - \alpha)^2}{2} \right]\\
   &= \sum_{i = 1}^n \left[- log\left( \sqrt{2 \pi}\right) - \left(\frac{y_i^2 + \alpha^2 - 2y_i\alpha}{2}\right) \right]\\
   &= \sum_{i = 1}^n \left[y_i \alpha - log\left( \sqrt{2 \pi}\right) - \left( \frac{y_i^2 + \alpha^2}{2} \right) \right]
\end{align*}
\end{example}

\begin{definition} [The Score Function]
\label{def:score_function}
Consider $\boldsymbol{\beta} = (\beta_1, \ldots, \beta_k)^T \in \Theta^k$, and assume that $\Theta^k$ is an open subspace of $\mathbb{R}^k$, and that the log-likelihood is continuously differentiable. Then  the following first order partial derivative of the log-likelihood function
\begin{align*}
    S(\boldsymbol{\beta}; \textbf{y}) = \ell'_{\boldsymbol{\beta}}(\boldsymbol{\beta}; \textbf{y}) = \frac{\partial}{\partial \boldsymbol{\beta}} \ell (\boldsymbol{\beta}; \textbf{y}) = 
    \begin{pmatrix}
        \frac{\partial}{\partial \theta_1}\ell (\boldsymbol{\beta}; \textbf{y}) \\
        \vdots \\
        \frac{\partial}{\partial \theta_k}\ell (\boldsymbol{\beta}; \textbf{y})
    \end{pmatrix}
\end{align*}
is the score function.
\end{definition}

\begin{example}
Consider again the model from example \ref{ex:model1} where the log likelihood function was derived
\begin{align*}
   \ell(\textbf{Y};\alpha) = \sum_{i = 1}^n \left[y_i \alpha - log\left( \sqrt{2 \pi}\right) - \left( \frac{y_i^2 + \alpha^2}{2} \right) \right].
\end{align*}

The score function is the log likelihood function differentiated with respect to its parameter, thus

\begin{align*}
    S\left( \textbf{Y}; \alpha \right) &= \ell'(\textbf{Y}; \alpha)\\
    &= \sum_{i=1}^n \left[ y_i + 0 - \alpha \right]\\
    &= \sum_{i=1}^n \left[ y_i - \alpha \right].
\end{align*}

This is the score function.

\end{example}

\begin{theorem}
Under the same conditions as definition \ref{def:score_function}
\begin{align*}
    E_{\boldsymbol{\beta}}[S(\boldsymbol{\beta}; \textbf{y})] = \textbf{0}
\end{align*}
\end{theorem}

\begin{definition} [Observed Information]
\label{def:observed_information}
The matrix
\begin{align} \label{eq:Observed_information}
    \textbf{j}(\boldsymbol{\beta};\textbf{y}) = - \frac{\partial^2}{\partial \boldsymbol{\beta} \partial \boldsymbol{\beta}^T} \ell(\boldsymbol{\beta}; \textbf{y})
\end{align}
with elements
\begin{align*}
    \textbf{j}(\boldsymbol{\beta};\textbf{y})_{ij} = - \frac{\partial^2}{\partial \beta_i \partial \beta_j} \ell(\boldsymbol{\beta}; \textbf{y})
\end{align*}
is called the observed information corresponding to the observations $\textbf{y}$ and the parameter $\boldsymbol{\beta}$.
\end{definition}

\begin{example}
    Consider again the situation from example \ref{ex:model1}.
    
    \begin{align*}
        j\left( Y; \alpha \right) &= - S\left(Y; \alpha \right)\\
        &=  - \left( \sum_{i = 1}^n -1\right) = n.
    \end{align*}
\end{example}

\begin{definition} [Expected Information]
\label{def:expected_information}
The expectation of the observed information 
\begin{align}
    i(\boldsymbol{\beta}) = E[\textbf{j}(\boldsymbol{\beta};\textbf{Y})],
\end{align}
where $\textbf{j}(\boldsymbol{\beta};\textbf{Y})$ is given by equation \eqref{eq:Observed_information}, and where the expectation is determined under the distribution corresponding to $\boldsymbol{\beta}$, is called the expected information matrix corresponding to the parameter $\boldsymbol{\beta}$.
\end{definition}

\begin{example}
    \begin{align*}
        i\left(\alpha \right) &= E\left[j(Y;\alpha)\right]\\
        &= n.
    \end{align*}
\end{example}

\begin{lemma} [Fisher Information Matrix]
\label{lem:fisher_information_matrix}
Under regularity conditions the expected information matrix is equal to the covariance-matrix for the score function
\begin{align*}
    \textbf{i}(\boldsymbol{\beta}) &= E_{\beta}\left[- \frac{\partial^2}{\partial \boldsymbol{\beta} \partial \boldsymbol{\beta}^T} \ell(\boldsymbol{\beta}; \textbf{y})\right] \\
    &= E_{\beta}\left[ \frac{\partial}{\partial \boldsymbol{\beta}}\ell (\boldsymbol{\beta}; \textbf{y}) \left( \frac{\partial}{\partial \beta} \ell (\boldsymbol{\beta}; \textbf{y}) \right) \right] \\
    &= D_{\beta} [\ell_\beta ' (\boldsymbol{\beta}; \textbf{y})],
\end{align*}
where $D_\beta[\cdot]$ denotes the covariance-matrix. 
\end{lemma}

\begin{definition} [Maximum Likelihood Estimate (MLE)]
\label{def:MLE}
Given an observation $Y=y$, the maximum likelihood estimate (MLE), $\hat{\boldsymbol{\beta}}(y))$ is said to exist if it is the unique maximum of the (log-)likelihood function. 
Let $E = \{ y : \hat{\boldsymbol{\beta}}(y) exists \}$. If $P_{\boldsymbol{\beta}}(Y \in E) = 1$ for all $\boldsymbol{\beta} in \Theta^k$, then $\hat{\boldsymbol{\beta}}(y))$ is called the maximum likelihood estimator (ML estimator).
\end{definition}
Note that the MLE is a solution to the ML equation
\begin{align*}
    S(\boldsymbol{\beta}; \textbf{y}) = 0
\end{align*}
It is therefore only a critical point and should always be checked, whether it is also a maximum point. 

\begin{theorem}[Distribution of the ML estimator]\label{th:distribution_ml_estimator}
Let $E = \{\mathbf{y} \ : \ \text{the MLE } \hat{\theta}(\mathbf{y}) \text{ exists}\}$. 
If $A$ is a $k \times k$ matrix, $A^{1/2}$ denotes the $k \times k$ matrix such that $A = A^{1/2}\left( A^{1/2} \right)^T$.
Let $\rightarrow^\mathcal{D}$ denote convergence in distribution, $\mathbf{1}[\cdot]$ the indicator function and $I_k$ the $k \times k$ identity matrix.
Then under regularity conditions, if $\mathbf{Y} \sim f(\cdot;\theta)$ then as $n \rightarrow \infty$ we have that
\begin{enumerate}
    \item $P_\theta(Y \in E) \rightarrow 1$
    \item for any $\varepsilon > 0: \ P_\theta(Y \in E, \ \|\hat{\theta} - \theta\| \leq 1) \rightarrow 1$
    \item $1[\mathbf{y} \in E] i(\hat{\theta})^{1/2}(\hat{\theta} - \theta) \rightarrow^\mathcal{D} N_k(0, I_k)$ and $1[\mathbf{y} \in E] j(\hat{\theta})^{1/2}(\hat{\theta} - \theta) \rightarrow^\mathcal{D} N_k(0, I_k)$
    \item $1[\mathbf{y} \in E] i(\theta)^{1/2}(\hat{\theta} - \theta) \rightarrow^\mathcal{D} N_k(0, I_k)$
\end{enumerate}
\end{theorem}

\begin{definition} [The (1-a)-confidence Region]
\label{def:confidence_region}
Suppose that for a given number $a \in (0,1)$ and for each $\textbf{y} \in \mathcal{Y}$, we have specified a subset $A(\textbf{y})$ of the parameter space $\Theta^k$, such that
\begin{align*}
    P_{\boldsymbol{\beta}} (\boldsymbol{\beta} \in A(\textbf{Y})) = 1 - a, \quad \forall \boldsymbol{\beta} \in \Theta^k
\end{align*}
Then we call $A(y)$ a $(1-a)$-confidence region for $\boldsymbol{\beta}$.
\end{definition}

